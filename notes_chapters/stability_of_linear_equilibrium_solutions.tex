\section*{Stability of linear equilibrium solutions}

\textbf{repeat Nevin's analysis here, and derive the growth rates for the daughters' non-linear amplitudes around the linear equilibrium.}

\begin{subequations}
\begin{align}
\dot{q}_1 + (i\omega_1 + \gamma_1)q_1 = & i\omega_1 U_1(t) + 2i\omega_1\kappa q_0^\ast q_2^\ast \\
\dot{q}_2 + (i\omega_2 + \gamma_2)q_2 = & i\omega_2 U_2(t) + 2i\omega_2\kappa q_0^\ast q_1^\ast \\
\end{align}
\end{subequations}

we assume there is a linear solution and consider perturbations about that assuming $q_0$ is given by it's linear value.

\begin{subequations}
\begin{align}
\dot{q}_{1,NL} + (i\omega_1 + \gamma_1)q_{1,NL} = & 2i\omega_1\kappa q_0^\ast q_{2,NL}^\ast \\
\dot{q}_{2,NL} + (i\omega_2 + \gamma_2)q_{2,NL} = & 2i\omega_2\kappa q_0^\ast q_{1,NL}^\ast \\
\end{align}
\end{subequations}

If we make the (standard) transformation $q_i \rightarrow x_i$, we can re-write these equations so they are time independent ($x_o$ is just a complex constant).

\begin{subequations}
\begin{align}
\dot{x}_{1,NL} + (i\Delta_1 + \gamma_1)x_{1,NL} = & 2i\omega_1\kappa x_0^\ast x_{2,NL}^\ast \\
\dot{x}_{2,NL} + (i\Delta_2 + \gamma_2)x_{2,NL} = & 2i\omega_2\kappa x_0^\ast x_{1,NL}^\ast \\
\end{align}
\end{subequations}

We further assume the solution is linear, in that it takes the form $\vec{x} \propto e^{st}$, which yields

\begin{subequations}
\begin{align}
(i\omega_1 + s+\Delta_1)x_{1,NL} = & 2i\omega_1\kappa x_0^\ast x_{2,NL}^\ast \\
(i\omega_2 + s+\Delta_2)x_{2,NL} = & 2i\omega_2\kappa x_0^\ast x_{1,NL}^\ast \\
\end{align}
\end{subequations}

Which we can combine as follows:
\begin{subequations}
\begin{align}
(i\Delta_1 + s+\gamma_1)(-i\Delta_2 + s+\gamma_2)x_{1,NL} = & (2i\omega_1\kappa x_0^\ast) ((-i\Delta_2 + s+\gamma_2)x_{2,NL}^\ast) \\
 = & (2i\omega_1\kappa x_0^\ast) (-2i\omega_2\kappa x_0 x_{1,NL}) \\
 = & -4\omega_1\omega_2 \kappa^2 |x_0|^2 x_{1,NL} \\
\Rightarrow (i\Delta_1 + s+\gamma_1)(-i\Delta_2 + s+\gamma_2) = & -4\omega_1\omega_2 \kappa^2 |x_0|^2
\end{align}
\end{subequations}

Which is quadratic in $s$, and there is a required symmetry $1 \leftrightarrow 2$, so we should expect 4 (complex) solutions to two quadratic equations. Maintaining our present course, we can solve for $s$ explicitly, which yields

\begin{subequations}\label{equation:3mode-eig}
\begin{align}
s = & \frac{1}{2}\left[ -(\gamma_1 + \gamma_2) -i(\Delta_1 - \Delta_2) \pm \sqrt{ (\gamma_1 + \gamma_2 + i(\Delta_1 - \Delta_2))^2 -4((\gamma_1 + i\Delta_1)(\gamma_2-i\Delta_2) - 4\omega_1\omega_2\kappa^2 A_0^2) } \right]\\
  = & \frac{1}{2}\left[ -(\gamma_1 + \gamma_2) -i(\Delta_1 - \Delta_2) \pm \sqrt{ (\gamma_1 - \gamma_2)^2 - (\Delta_1 + \Delta_2)^2 + + 16\omega_1\omega_2\kappa^2A_0^2 + 2i(\Delta_1+\Delta_2)(\gamma_1-\gamma_2)  } \right]
\end{align}
\end{subequations}

and there is an implicit set of companion solutions: $1\leftrightarrow 2$. We're interested in the solutions for which $\mathbb{R}\{s\} \geq 0$, which are (marginally) unstable to linear perturbations. While this takes some algebraic back-flips to accomplish, it is relatively straightforward to derive 

\begin{subequations}
\begin{align}
\mathbb{R}\{s\} = \frac{1}{2} & \left[ -(\gamma_1+\gamma_2) \pm \left( ( (\gamma_1-\gamma_2)^2 - (\Delta_1+\Delta_2)^2 + 16\omega_1\omega_2\kappa^2A_0^2)^2  + 4(\gamma_1-\gamma_2)^2(\Delta_1+\Delta_2)^2 \right)^{1/4} \right.\\
                              & \left.\ \times\cos\left( \frac{1}{2} \tan^{-1} \left( \frac{2(\Delta_1+\Delta_2)(\gamma_1-\gamma_2)}{(\gamma_1-\gamma_2)^2 - (\Delta_1+\Delta_2)^2 + 16\omega_1\omega_2\kappa^2A_0^2 }\right)\right)     \right] 
\end{align}
\end{subequations}

Which looks to be transcendental. However, if we solve for the marginally stable case $\mathbb{R}\{s\} = 0$, and square both sides we can apply a trig-identity $cos^2(\theta) = (1+cos(2\theta))/2$, which renders the problem algebraic. Further manipulation reveals the threshold energy for the parent mode:

\begin{subequations}\label{equation:3mode Athr}
\begin{align}
(A_0^2)_{thr} = & \frac{(\gamma_1+\gamma_2)^2 - (\gamma_1-\gamma_2)^2}{(\gamma_1+\gamma_2)^2}\left[\frac{1}{16\omega_1\omega_2\kappa^2}\left( (\gamma_1+\gamma_2)^2 + (\Delta_1+\Delta_2)^2\right) \right] \\
              = & \frac{\gamma_1\gamma_2}{4\omega_1\omega_2\kappa^2}\left( 1 + \frac{(\Delta_1+\Delta_2)^2}{(\gamma_1+\gamma_2)^2}\right) \\
\end{align}
\end{subequations}

which is the exact solution for the 3-mode stability threshold.

\subsection*{assympotic limits for 3-mode eigenvalues: $A_0^2 \gg A_{thr}^2$}

Taking as our staring point Equation \ref{equation:3mode-eig}, we define 

\begin{equation}
\epsilon \equiv \frac{A_thr}{A_0} \ll 0
\end{equation}

Upon substition, this yields

\begin{subequations}
\begin{align}
2s + (\gamma_1+\gamma_2) + i(\Delta_1-\Delta_2) = & \pm \sqrt{ (\gamma_1 - \gamma_2)^2 - (\Delta_1 + \Delta_2)^2 + + 16\omega_1\omega_2\kappa^2A_0^2 + 2i(\Delta_1+\Delta_2)(\gamma_1-\gamma_2)  }  \\
\Rightarrow 2s + (\gamma_1+\gamma_2) + i(\Delta_1-\Delta_2) = & \pm \sqrt{ (\gamma_1 - \gamma_2)^2 - (\Delta_1 + \Delta_2)^2 + 16\omega_1\omega_2\kappa^2A_{thr}^2\frac{1}{\epsilon^2} + 2i(\Delta_1+\Delta_2)(\gamma_1-\gamma_2)  } \\
 = & \pm \frac{1}{\epsilon} 4\sqrt{\omega_1\omega_2}\kappa A_{thr} \sqrt{ 1 + \frac{(\gamma_1 - \gamma_2)^2 - (\Delta_1 + \Delta_2)^2 + 2i(\Delta_1+\Delta_2)(\gamma_1-\gamma_2)}{16\omega_1\omega_2\kappa^2A_{thr}^2 }\epsilon^2 }  \\
 = & \pm \frac{1}{\epsilon} 4\sqrt{\omega_1\omega_2}\kappa A_{thr} \left( 1 + \frac{1}{2}\frac{(\gamma_1 - \gamma_2)^2 - (\Delta_1 + \Delta_2)^2 + 2i(\Delta_1+\Delta_2)(\gamma_1-\gamma_2)}{16\omega_1\omega_2\kappa^2A_{thr}^2 }\epsilon^2 + O(\epsilon^4)\right)  \\
 = & \pm \left( \frac{1}{\epsilon} 4\sqrt{\omega_1\omega_2}\kappa A_{thr}  + \frac{1}{2}\frac{(\gamma_1 - \gamma_2)^2 - (\Delta_1 + \Delta_2)^2 + 2i(\Delta_1+\Delta_2)(\gamma_1-\gamma_2)}{4\sqrt{\omega_1\omega_2}\kappa A_{thr} }\epsilon + O(\epsilon^3)\right)  \\
\end{align}
\end{subequations}

from which we can immediately write

\begin{subequations}
\begin{align}
\mathbb{R}\{s\} = & -\frac{1}{2}(\gamma_1+\gamma_2) \pm \left( 2\sqrt{\omega_1\omega_2}\kappa \frac{A_{thr}}{\epsilon} + \frac{(\gamma_1-\gamma_2)^2 - (\Delta_1+\Delta_2)^2}{8\sqrt{\omega_1\omega_2}\kappa} \frac{\epsilon}{A_{thr}} + O(\epsilon^3) \right) \\
                = & -\frac{1}{2}(\gamma_1+\gamma_2) \pm 2\sqrt{\omega_1\omega_2}\kappa A_0 + O(\epsilon) \\
          \approx & \pm 2\sqrt{\omega_1\omega_2}\kappa A_0 + O(\epsilon) \\
\end{align}
\end{subequations}

and 

\begin{subequations}
\begin{align}
\mathbb{I}\{s\} = & \frac{1}{2}(\Delta_1-\Delta_2) \pm \frac{(\Delta_1-\Delta_2)*(\gamma_1-\gamma_2)}{8\sqrt{\omega_1\omega_2}\kappa}\frac{\epsilon}{A_{thr}} + O(\epsilon^3) \\
          \approx & \frac{1}{2}(\Delta_1-\Delta_2) + O(\epsilon) 
\end{align}
\end{subequations}

Note that there are 4 total eigenvalues (the $\pm$ and $2\leftrightarrow1$ exchange), but there are only 2 distinct $\mathbb{R}\{s\}$ values and 2 distince $\mathbb{I}\{s\}$ values, corresponding to growing and decaying modes, each able to oscillate in both directions. Furthermore, the quantity $\Delta_1+\Delta_2=\Omega+\omega_1+\omega_2$ is determined by out transformation and there is no ambiguity in $\mathbb{R}\{s\}$. However, $\Delta_1-\Delta_2$ is not determined and there is an ambiguity in the oscillation frequencies. Because we are mostly interested in the growth rates, this is acceptable. 

It appears that this ambiguity disappears in integration data and modes typically oscillate at $\Omega/2$. A hand-wavey argument for this is as follows. Assume the daughter modes obey

\begin{equation}
\dot{q}_c + (i\omega_c+\gamma_c)q_c = \omega_c \sum_b \kappa_{0bc} q_0^\ast q_b^\ast = \omega_c \sum_b \kappa_{0bc} A_0 e^{i m_0 \Omega_{orb}t} q_b^\ast | \dot{A_0} = 0
\end{equation}

If we assume the daughter behave as $q_c = A_c e^{s_c t} | \dot{A}_c=0$, then we have

\begin{subequations}
\begin{align}
(s_c + i\omega_c + \gamma_c) A_c e^{s_c t} = & \omega_c \sum_b \kappa_{0bc} A_0 A_b e^{(s_b^\ast + im_0\Omega_orb)t} \\
(s_c + i\omega_c + \gamma_c) A_c = & \omega_c \sum_b \kappa_{0bc} A_0 A_b e^{(s_b^\ast-s_c - i(m_b+m_c)\Omega_orb)t} \\
\end{align}
\end{subequations}

because $m_a+m_b+m_c=0$, as required by the coupling. The LHS is a constant, so we need to remove all time dependence from the RHS, meaning

\begin{subequations}
\begin{align}
s_b^\ast - s_c - i(m_b+m_c)\Omega_{orb} = & 0 \ \forall \ b,c \\
\Rightarrow \mathbb{R}\{s_b\} = & \mathbb{R}\{s_c\} \ \forall \ b,c \\
 \mathbb{I}\{s_b\} + \mathbb{I}\{s_c\} = & -(m_b+m_c)\Omega_{orb} \ \forall \ b,c\\
\end{align}
\end{subequations}

The last constraint (on the imaginary parts of the eigenfunctions), can be satisfied in two way:
\begin{enumerate}
  \item{$\mathbb{I}\{s_b\} = -m_b\Omega_{orb} \ \forall \ b$}
  \item{$\mathbb{I}\{s_b\} = \frac{1}{2}m_a\Omega_{orb} \ \forall \ b$}
\end{enumerate}

Integration appers to favor the latter, most likely because it is more symmetric between the daughter modes (and therefore more stable?).


\subsection*{assymptotic limits for 3-mode eigenvalues: $A_0^2 - A_{thr}^2 \ll A_{thr}^2$}

Here, we define

\begin{equation}
\epsilon = \frac{A_0^2}{A_{thr}^2} - 1 \ll 0
\end{equation}

which yields

\begin{subequations}
\begin{align}
2s +(\gamma_1+\gamma_2)+i(\Delta_1-\Delta_2) = & \pm \sqrt{ (\gamma_1-\gamma_2)^2 - (\Delta_1+\Delta_2)^2 + 16\omega_1\omega_2\kappa^2 A_{thr}^2 (1+\epsilon) +  2i(\Delta_1+\Delta_2)(\gamma_1+\gamma_2) } \\
= & \sqrt{ (\gamma_1-\gamma_2)^2 - (\Delta_1+\Delta_2)^2 + 16\omega_1\omega_2\kappa^2 A_{thr}^2 +  2i(\Delta_1+\Delta_2)(\gamma_1+\gamma_2) } \\
  &\ \times \left(1 + \frac{1}{2}\frac{16\omega_1\omega_2\kappa^2 A_{thr}^2}{(\gamma_1-\gamma_2)^2 - (\Delta_1+\Delta_2)^2 + 16\omega_1\omega_2\kappa^2 A_{thr}^2 +  2i(\Delta_1+\Delta_2)(\gamma_1+\gamma_2)} \epsilon  + O(\epsilon^3) \right) \\
\end{align}
\end{subequations}

which is horrible. However, by definition 

\begin{equation}
\mathbb{R}\{\sqrt{ (\gamma_1-\gamma_2)^2 - (\Delta_1+\Delta_2)^2 + 16\omega_1\omega_2\kappa^2 A_{thr}^2 +  2i(\Delta_1+\Delta_2)(\gamma_1+\gamma_2) } \} = (\gamma_1+\gamma_2)
\end{equation}

and we can simply define

\begin{equation}
\mathbb{I}\{\sqrt{ (\gamma_1-\gamma_2)^2 - (\Delta_1+\Delta_2)^2 + 16\omega_1\omega_2\kappa^2 A_{thr}^2 +  2i(\Delta_1+\Delta_2)(\gamma_1+\gamma_2) } \} \equiv \Theta
\end{equation}

choosing the maximum eigenvalue, of course. This allows us to write

\begin{subequations}
\begin{align}
2s +(\gamma_1+\gamma_2)+i(\Delta_1-\Delta_2) = &  ( (\gamma_1+\gamma_2) + i\Theta ) \left( 1 + \frac{1}{2}\frac{16\omega_1\omega_2\kappa^2 A_{thr}^2}{(\gamma_1+\gamma_2 + i\Theta)**2} \epsilon + O(\epsilon^2) \right)\\
\Rightarrow 2s + i(\Delta_1-\Delta_2 \pm \Theta) = &  \frac{1}{2}\frac{16\omega_1\omega_2\kappa^2 A_{thr}^2}{(\gamma_1+\gamma_2)**2 + \Theta**2} (\gamma_1+\gamma_2-i\Theta) \epsilon + O(\epsilon^2) \\
\end{align}
\end{subequations}

which we can simplify to
\begin{subequations}
\begin{align}
\mathbb{R}\{s\} = & \frac{1}{2}\frac{16\omega_1\omega_2\kappa^2 A_{thr}^2}{(\gamma_1+\gamma_2)**2 + \Theta**2} (\gamma_1+\gamma_2) \epsilon + O(\epsilon^2) \\
                = & \left[\frac{(\gamma_1+\gamma_2)\gamma_1\gamma_2 \left( 1 + \left(\frac{\Delta_1+\Delta_2}{\gamma_1+\gamma_2}\right)^2\right)}{ \left( \left[(\gamma_1-\gamma_2)^2 - (\Delta_1+\Delta_2)^2 + 4\gamma_1\gamma_2\left( 1 + \left(\frac{\Delta_1+\Delta_2}{\gamma_1+\gamma_2}\right)^2\right) \right]^2 + 4(\Delta_1+\Delta_2)^2(\gamma_1+\gamma_2)^2\right)^{-1/2} }\right] \epsilon + O(\epsilon^2) \\
\end{align}
\end{subequations}

where we've substituted 

\begin{equation}
16\omega_1\omega_2\kappa^2 A_{thr}^2 = \gamma_1\gamma_2 \left( 1 + \left(\frac{\Delta_1+\Delta_2}{\gamma_1+\gamma_2}\right)^2\right)
\end{equation}

and used the definition of $\Theta$ and it's relation to $(\gamma_1+\gamma_2)$ to deteriming the norm.

There is an analogous equation for the imaginary components, but that is unteresting. Note that the real part only depends on the combination $(\Delta_1+\Delta_2)=\Omega+\omega_1+\omega_2$ and there is no ambiguity in $\mathbb{R}\{s\}$. Furthermore, we should point out that the prefactor multiplying $\epsilon$ is almostly commically complicated and it should be no surprise to find cases where the minimum $A_{thr}$ pair does \emph{emph} correspond to the maximum $\mathbb{R}\{s\}$ pair. This is commonly seen in integrations and could be due to differences in this prefactor.


\subsection*{stability of single self-coupled daughter}

this probably isn't the correct place for this, but derive the stability/threshold energy for the parent here using:
\begin{equation}
\dot{q}_{d,NL} + (i\omega_d + \gamma_d)q_{d,NL} = 2i\omega_d \kappa q_0^\ast q_{d,NL}^\ast
\end{equation}

By similar arguments, we arrive at the same stability threshold and can simply replace $2\rightarrow1$ in the above solution.

